\documentclass{article}

\usepackage[utf8]{inputenc}
\usepackage[T1]{fontenc}
\usepackage{geometry}
\geometry{a4paper}
\usepackage{graphicx}
\DeclareUnicodeCharacter{2212}{\ensuremath{-}}
\title{APPM4600 Homework 6}
\author{Owen O'Connor}
\date{October 11th 2024}
\begin{document}
\maketitle

\section{Question 1}
See GitHub for full code
\subsection{i}
For initial guess [x, y] = [1, 1]:

The approximate solution using Broyden method is [-1.81626407  0.8373678 ] in 12 iterations
The approximate solution using the Lazy Newton method is [nan nan] in 499 iterations.

\subsection{ii}
For initial guess [x, y] = [1, -1]:

The approximate solution using Broyden method is [ 1.00416874 -1.72963729] in 6 iterations
The approximate solution using the Lazy Newton method is [ 1.00416874 -1.72963729] in 36 iterations

So the Broyden method is way faster than Lazy Newton, which is expected since you don't have to compute the Jacobian.
\subsection{iii}
For initial guess $x = 0$, $y = 0$, the Jacobian is singular so it's impossible to solve via Lazy Newton or Broyden.

\section{Question 2}
Using initial guess of [x, y, z] = [0.5, 0.5, 0.5] 

The approximate solution using the Steepest Descent method is [0.00304809 0.1029706  1.00086192] in 5 iterations.

Using this as a new initial guess and then applying Newton's method, the approximate solution is [-5.22760036e-08  9.99986414e-02  9.99989812e-01] in 1 iteration.

So Newton's converges much faster.








\end{document}
